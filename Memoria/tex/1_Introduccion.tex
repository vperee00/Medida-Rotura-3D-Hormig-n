\capitulo{1}{Introducción}

El estudio de los materiales de construcción como el hormigón, que se emplea en todo tipo de estructuras y edificaciones como puentes, túneles y distintos tipos de edificios, es una parte fundamental para conocer los límites y el comportamiento de los mismos. En el caso concreto del hormigón, características como su  durabilidad o su resistencia a los impactos ha hecho que sea ampliamente usado como material de construcción. Pero otra de las propiedades de este material es su fragilidad, lo que le hace propenso a que aparezcan grietas debido a distintos factores como pueden ser las variaciones de temperatura, la humedad o la presión a la que se es sometido en las distintas estructuras de las que forma parte. \cite{Zhang2024Review} Por ello, es necesario la creación de métodos que permitan medir sus propiedades para poder predecir el comportamiento del mismo, adecuar las construcciones a estas propiedades, y que estos métodos permitan la revisión y detección de posibles problemas a lo largo de la vida de las construcciones.

El hormigón es el material de construcción más utilizado a nivel mundial debido a su versatilidad, durabilidad y buen comportamiento frente a la compresión. Sin embargo, se trata de un material de naturaleza cuasi-frágil, lo que implica que no falla de manera súbita como un material puramente frágil, sino que presenta un proceso de fisuración progresiva previo al colapso. Esta característica hace que la aparición y evolución de fisuras en elementos de hormigón sea un fenómeno inevitable y, al mismo tiempo, de gran relevancia desde el punto de vista estructural. El control de dichas fisuras es fundamental para garantizar la seguridad y prolongar la vida útil de las infraestructuras. Las fisuras no solo reflejan el estado tensional y el nivel de daño acumulado en el material, sino que también constituyen vías de entrada de agentes externos, como el agua o los cloruros, que pueden alcanzar la armadura embebida en el hormigón. La corrosión del acero, favorecida por la penetración de dichos agentes, reduce la capacidad resistente de la estructura y puede desembocar en fallos graves. Ejemplos recientes, como el colapso del puente Morandi en Génova o del puente Carola en Dresde, evidencian las consecuencias potencialmente catastróficas de una degradación no controlada. Por este motivo, resulta esencial disponer de sistemas eficaces que permitan medir fisuras de forma fiable y objetiva. Estos sistemas deben ser capaces de detectar su aparición, cuantificar su anchura y determinar su longitud, proporcionando así información clave para implementar estrategias de mantenimiento preventivo y garantizar la seguridad estructural a largo plazo.

Una alternativa de creciente interés para mejorar el comportamiento del hormigón es la incorporación de fibras de acero, que permiten controlar la fisuración y aumentar la capacidad post-fisuración del material. Este refuerzo resulta especialmente relevante en elementos sometidos a esfuerzos de flexión, donde el hormigón tiende a fisurarse de manera temprana debido a su baja resistencia a tracción. Para caracterizar el comportamiento del hormigón reforzado con fibras de acero se emplea habitualmente el ensayo de flexión en tres puntos descrito en la norma EN 14651. En este ensayo, el parámetro más representativo es la apertura de fisura o Crack Mouth Opening Displacement (CMOD), que se mide normalmente mediante extensómetros de clip situados en la entalla de la probeta. Aunque este sistema proporciona resultados precisos y repetibles en condiciones de laboratorio, presenta limitaciones importantes a la hora de ser aplicado en estructuras reales: el equipamiento es costoso, la instrumentación resulta laboriosa y, además, exige conocer con antelación la posición exacta donde se iniciará la fisura. Estas restricciones han impulsado la investigación en métodos alternativos para el control de fisuración, entre los cuales destacan los basados en análisis de imagen, capaces de ofrecer una solución más flexible y potencialmente escalable a la monitorización de estructuras en servicio.

Los tres tipos de métodos principales de medición y análisis usados han sido \cite{MenaAlonso2023} la medición manual, la medición a través de distintos tipos de sensores, y el análisis de imágenes.

Los métodos manuales consisten en que un técnico especialista revise de forma visual las estructuras en búsqueda de la posible aparición de grietas. Hacen uso de distintos tipos de patrones y esquemas para poder analizar el posible origen de la grieta y cómo afecta al conjunto de la estructura, apoyándose en mediciones sobre la anchura de las grietas. La principal ventaja de este método es su bajo coste y su simplicidad, con lo que han sido los más utilizados de forma tradicional para realizar las tareas de mantenimiento de las estructuras de hormigón. Por el contrario, su mayor desventaja es que son mediciones subjetivas que dependen del conocimiento y la experiencia del técnico que las realiza.

Las mediciones a través de sensores se utilizan para solventar este problema. Hay diversos tipos de sensores y de tipos de mediciones como ultrasonidos \cite{Lootens2020Continuous}, medidores de tensión, sensores acústicos, extensómetros de clip, de infrarrojos, por nombrar unos pocos ejemplos. Todos estos conjuntos de sensores permiten realizar medidas objetivas para analizar la aparición de las grietas y analizar el estado actual de los materiales. Por otro lado, las desventajas que presentan es su elevado coste y que sólo pueden realizar las medidas en puntos concretos, lo que hace dificil usarlos para estudiar la evolución de las grietas mientras estas se producen.

El método de análisis de imágenes intenta solventar los problemas de coste y de seguimiento de únicamente de puntos concretos de los sensores. Para ello, se utilizan diferentes técnicas de análisis de imagenes \cite{Khan2023Image} como la transformada de wavelet, la correlación digital de imágenes (DIC), métodos de percolación, detección de bordes Canny, bosques aleatorios, redes neuronales convolucionales... Estos métodos utilizan una serie de imágenes capturadas a lo largo del tiempo para analizar un conjunto de puntos y su desplazamiento a lo largo del tiempo de la captura. Además de ser más flexibles en cuanto a su colocación para la realización de las capturas de imágenes que los sensores, también son menos invasivos debido a que esa captura se puede realizar sin tener que alterar el material sobre el que se está realizando la medición.

En este trabajo se ha utilizado este último método. Se ha contado con un conjunto de pares de imágenes capturadas mientras se sometía a una probeta prismática de hormigón reforzado con fibras de acero a un ensayo de flexión en tres puntos de acuerdo con la norma EN14651. Estos pares de imágenes correspondían a dos cámaras de tal forma que fuera posible a través de la diferencia de perspectiva, usar la visión estereoscópica para realizar una representación en tres dimensiones de la escena. Para este proceso se han utilizado modelos basados en inteligencia artificial, que nos han permitido obtener la posición en el espacio de los píxeles de las imágenes. Y a partir de esa posición, conocer su profundidad y combinar estas dos medidas para poder detectar donde se ha producido una grieta y medir su tamaño.
