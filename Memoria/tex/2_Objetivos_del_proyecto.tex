\capitulo{2}{Objetivos del proyecto}

Los objetivos del proyecto son:

\begin{enumerate}
	\item A partir de pares de imágenes, obtener una representación 3D del bloque de hormigón.
	\item Una vez obtenida una representación en el espacio tridimensional, analizar si las grietas son detectadas y representadas en el modelo en 3D.
	\item A partir del modelo 3D con la grietas, poder realizar una medición del tamaño de las mismas.
	\item Crear un método automatizado para la medida de las grietas que permita el estudio de su evolución.
\end{enumerate}

Para ello se han probado distintos modelos de inteligencia artifical, que nos permiten obtener una posición en un espacio tridimensional de los píxeles de las imágenes obtenidas. Y usando los datos de esas representaciones, como las posiciones de los puntos y su profundidad, poder detectar dónde se ha producido una grieta y cuál es su tamaño.